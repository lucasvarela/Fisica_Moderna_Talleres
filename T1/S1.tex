  \documentclass[12pt]{article}
 
\usepackage[margin=1in]{geometry}
\usepackage{amsmath,amsthm,amssymb}
\usepackage[spanish]{babel}
\usepackage[usenames, dvipsnames]{color}
\decimalpoint
\usepackage[utf8]{inputenc}
\usepackage{enumitem, kantlipsum}
\usepackage{graphicx}
\setlength{\parindent}{0cm} 

\begin{document}
 
 \color{ForestGreen}
\begin{center}
\Large \textbf{C.Física Moderna: Solución Taller 1}
%\large \textbf{C.Física Moderna: Taller 1}
\end{center}
 

 \textbf{Para todos los ejercicios:} aproxime la rapidez de la luz a $c=3\times 10^8$ m/s.\\ 
\noindent\rule{16.5cm}{0.4pt}\\ 
 \textbf{1. Calentamiento}\\
 
  Para este ejercicio se usa $x,t$ como las coordenadas respecto al marco $S$ y $x',t'$ respecto a $S'$. El marco $S'$ se mueve con velocidad $v$ relativa a $S$, en la dirección del eje $x$. Los orígenes de ambos marcos coinciden en el tiempo $t = t' = 0$.\\
 
 \textbf{1.1 Transformaciones de Lorentz}
 
 Para este problema (solo el 1.1) tome $v = 0.6c$. Encuentre las coordenadas de los siguientes eventos para $S'$:
 
 \begin{enumerate}
 	\item $x=4$m, $t = 0$s\\
 	\item $x=4$m, $t = 1$s\\
 	\item $x=1.8\times 10^8$m, $t = 1$s\\
 	\item $x=10^9$m, $t = 2$s\\
 \end{enumerate} 
 
 

\textbf{1.2 Velocidad relativa entre marcos de referencia}\\

Considere un evento que ocurre en la posición $x=6\times 10^8$m, $x'=6\times 10^8$m  y $t'=4$s. ¿Cual es la \textbf{velocidad} relativa entre los marcos $S$ y $S'$? 


\noindent\rule{16.5cm}{0.4pt}


 \color{Black}
 
 \textbf{1.1 Transformaciones de Lorentz}
 
 Para calcular las cantidades que se piden se hace utilizando las transformaciones de Lorentz respectivas:
 
 \begin{align}
 x' = \gamma (x-vt)\\
 t' = \gamma \left(t- \frac{vx}{c^2}\right)
 \end{align}
 
 Reemplazando se obtiene:
 
 \begin{enumerate}
 	\item $x'=5$m , $t'=-10^{-8}$s
	\item $x'=-2.225\times 10^{8}$m , $t'=1.25$s
	\item $x'=0$m , $t'=0.8$s
 	\item $x'=8 \times 10^8$m , $t'=0$s
 \end{enumerate}

\noindent\rule{16.5cm}{0.4pt}\\ 
\textbf{1.2 Velocidad relativa entre marcos de referencia}\\

La transformación de Lorentz para la posición es:

 
 \begin{equation}
 x  = \frac{x'+  c \beta  t'}{\sqrt{1-\beta^2}}
 \end{equation}
 
 Dado que $x = x'$, se obtiene una ecuación para $v$:
 
 \begin{equation}
 x'+  v  t' = \sqrt{1-(v/c)^2} x'
 \end{equation}
 
 Elevando al cuadrado ambos lados:
 
 
  \begin{equation}
 x'^2+ 2x'v  t' +(v  t')^2 = (1-(v/c)^2)x'^2
 \end{equation}
 
 Despejando se obtiene:
 
  \begin{equation}
 v\left( v\left( \frac{x'^2}{c^2} + t'^2\right) + 2x't'\right) = 0
 \end{equation}
 
 De donde se obtienen las dos velocidades relativas que pueden tener: $v=0$ o $v=-2.4\times 10^8$ m/s.\\ 
 
 
 
 \color{ForestGreen}
\noindent\rule{16.5cm}{0.4pt}\\
\textbf{2. Dilatación}

Dos eventos ocurren en el mismo lugar para $S'$ pero separados 4 segundos uno del otro. Sabiendo que en $S$ estos eventos están separados 6 segundos uno del otro, determine su separación espacial(en $S$).\\
\noindent\rule{16.5cm}{0.4pt}
\color{Black}

El tiempo se dilata por lo que:

\begin{equation}
\gamma = 6/4 = 3/2
\end{equation}

Por lo que $\beta  = \sqrt{5}/3$. 

Con esto podemos calcular la distancia:

\begin{equation}
x = \frac{\sqrt{5}c }{3} 6\text{s} = 1.341 \times 10^9 \text{m}  
\end{equation}

Los eventos están separados $ 1.341 \times 10^9 \text{m}$. Este problema se puede resolver aun más fácil usando el invariante $\Delta s$, inténtelo!
\color{ForestGreen}

\noindent\rule{16.5cm}{0.4pt}


\textbf{3. Tren}\\

 Un tren con longitud propia $L$  se mueve a velocidad constante con una rapidez de $c/2$ con
 respecto al piso. Una bola se tira desde la parte trasera hacia el frente del tren, con una velocidad de $c/3$ con respecto al tren. Determine el tiempo que tarda en recorrer el tren y la distancia que cubre visto desde:

 \begin{enumerate}
	\item  El marco de referencia del tren.
	\item  El marco de referencia del piso.
	\item  El marco de referencia de la bola.
	\item  Calcule el invariante “$(\Delta s)^2$ ” para cada marco de referencia. 
	$$(\Delta s)^2 = c^2(\Delta t)^2 - (\Delta x)^2$$
	\item  Muestre que los intervalos de tiempo en el marco de referencia de la bola y el piso están relacionados por el factor gamma relevante.
	\item  Muestre que los tiempos en el marco del tren y el piso no están relacionados por el factor relevante $\gamma$. ¿Por qué pasa esto?
\end{enumerate} 
\noindent\rule{16.5cm}{0.4pt}\\

\color{Black}

\textbf{3.1 El marco de referencia del tren.}\\


 En el marco del tren tenemos lo siguiente:

\begin{equation}
t = \frac{L}{c/3} = 3 \frac{L}{c}
\end{equation}

\begin{equation}
\Delta x = L 
\end{equation}

El tiempo es la distancia viajada(la cual es la longitud propia del tren) dividida por la rapidez de la bola.\\
\noindent\rule{16.5cm}{0.4pt}\\
\textbf{3.2 El marco de referencia del piso.}\\

 Usando la transformación de Lorentz tenemos:
 
 \begin{align*}
 t &= \gamma \left( t'+ \frac{v_x x'}{c^2}\right)\\
 x &= \gamma (x'+v_x t')
 \end{align*}
 
 En el marco del tren, $x' = L$ \& $t'  = 3L/c$.La velocidad relativa entre el piso y el tren es $v_x = c/2$, luego $\gamma = 2/ \sqrt{3}$.
 Por lo que:
 
 
  \begin{align*}
 \Delta t &= \frac{7L}{\sqrt{3} c}\\
 \Delta x &= \frac{5L}{\sqrt{3} }
 \end{align*}
\noindent\rule{16.5cm}{0.4pt}\\
\textbf{3.3 El marco de referencia de la bola.}\\


 Para el tiempo se usa la ecuación del caso anterior pero con $v_x = -c/3$ \& $\gamma = 3/ \sqrt{8}$. Luego
 
 
\begin{equation}
\Delta t = \sqrt{8} \frac{L}{c}
\end{equation}
\noindent\rule{16.5cm}{0.4pt}\\
\textbf{3.4 Los invariantes.}\\

 El intervalo invariante viene dado por la siguiente igualdad:

\begin{equation}
(\Delta s)^2 = c^2(\Delta t)^2 - (\Delta x)^2
\end{equation}
Usando de indice T,G,B para el tren, piso(“ground”) \& bola respectivamente:
\begin{align}
\Delta s_T = (3L)^2 - L^2& = 8 L^2\\
\Delta s_G= \left(\frac{7 L}{\sqrt{3}}\right)^2 - \left(\frac{5 L}{\sqrt{3}}\right)^2& = 8 L^2\\
\Delta s_B = (\sqrt{8}L)^2& = 8 L^2\\
\end{align}



\noindent\rule{16.5cm}{0.4pt}\\
\textbf{3.5 Gamma Bola-Piso}\\

La velocidad relativa entre el marco de la bola y el piso se calcula usando al formula de adición de velocidades:

\begin{equation}
u'= \frac{u+v}{1+ uv/c^2}
\end{equation}

Aquí  $v = c/2$ es la rapidez relativa entre el tren y el piso. $u = c/3$ es la rapidez de la bola vista desde el marco del
tren. Entonces $u' = 5/7$ es la velocidad relativa entre la bola y el piso. El factor gamma para $u'$ es $\gamma = 7/\sqrt{24}$. Entonces se tiene:


\begin{equation}
\Delta t_G = \sqrt{8} \frac{7L}{\sqrt{3}c} = \frac{7L}{\sqrt{3}c} \frac{\sqrt{8}}{\sqrt{8}} = \left(\frac{\sqrt{8}L}{c}\right) \left( \frac{7}{\sqrt{24}}\right) = \Delta t_B \gamma
\end{equation}
\noindent\rule{16.5cm}{0.4pt}\\
\textbf{3.6 No gamma Tren-Piso}\\

El factor gamma relevante esta dado por $\gamma = 2/ \sqrt{3}$ (como ya se mostró en (ii)). Para mostrar que los tiempos
no están relacionados por este factor tome el cociente $t_G /t_T$:

\begin{equation}
\frac{t_G}{t_T} = \frac{7}{3\sqrt{3}}\neq  \frac{2}{\sqrt{3}} = \gamma
\end{equation}

No están relacionados por el factor gamma! Esto es debido a que el factor gamma aparece solo cuando en uno de
los dos marcos de referencia los dos eventos ocurren en el mismo lugar, como ocurre en el caso (v) que la posición
nunca cambio.


\color{ForestGreen}


\noindent\rule{16.5cm}{0.4pt}


\end{document}