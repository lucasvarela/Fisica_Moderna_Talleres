  \documentclass[12pt]{article}
 
\usepackage[margin=1in]{geometry}
\usepackage{amsmath,amsthm,amssymb}
\usepackage[spanish]{babel}
\decimalpoint
\usepackage[utf8]{inputenc}
\usepackage{enumitem, kantlipsum}
\usepackage{graphicx}
\setlength{\parindent}{0cm} 

\begin{document}
 
\begin{center}
\Large \textbf{C.Física Moderna: Taller 1}\\
\normalsize \textbf{Transformaciones de Lorentz, adición de velocidades, contracción del espacio y dilatación del tiempo. }
\end{center}
 
 
 \textbf{Para todos los ejercicios:} aproxime la rapidez de la luz a $c=3\times 10^8$ m/s.\\ 
\noindent\rule{16.5cm}{0.4pt}\\ 
 \textbf{1. Calentamiento}\\
 
  Para este ejercicio se usa $x,t$ como las coordenadas respecto al marco $S$ y $x',t'$ a $S'$. El marco $S'$ se mueve con velocidad $v$ relativa a $S$, en la dirección del eje $x$. Los orígenes de ambos marcos coinciden en el tiempo $t = t' = 0$.\\
 
 \textbf{1.1 Transformaciones de Lorentz}
 
 Para este problema (solo el 1.1) tome $v = 0.6c$. Encuentre las coordenadas de los siguientes eventos para $S'$:
 
 \begin{enumerate}
 	\item $x=4$m, $t = 0$s\\
 	\item $x=4$m, $t = 1$s\\
 	\item $x=1.8\times 10^8$m, $t = 1$s\\
 	\item $x=10^9$m, $t = 2$s\\
 \end{enumerate} 
 
 

\textbf{1.2 Velocidad relativa entre marcos de referencia}\\

Considere un evento que ocurre en la posición $x=6\times 10^8$m, $x'=6\times 10^8$m  y $t'=4$s, ¿cuál es la \textbf{velocidad} relativa entre los marcos $S$ y $S'$? 


\noindent\rule{16.5cm}{0.4pt}



\textbf{2. Dilatación del tiempo}

Dos eventos ocurren en el mismo lugar para $S'$ pero separados 4 segundos uno del otro. Sabiendo que en $S$ estos eventos están separados 6 segundos uno del otro, determine su separación espacial(en $S$).


\noindent\rule{16.5cm}{0.4pt}


\textbf{3. Tren}\\

 Un tren con longitud propia $L$  se mueve a velocidad constante con una rapidez de $c/2$ con
 respecto al piso. Una bola se tira desde la parte trasera hacia el frente del tren, con una velocidad de $c/3$ con respecto al tren. Determine el tiempo que tarda en recorrer el tren y la distancia que cubre visto desde:

 \begin{enumerate}
	\item  El marco de referencia del tren.
	\item  El marco de referencia del piso sobre el que se mueve el tren.
	\item  El marco de referencia de la bola.
	\item  Calcule el invariante “$(\Delta s)^2$ ” para cada marco de referencia. 
	$$(\Delta s)^2 = c^2(\Delta t)^2 - (\Delta x)^2$$
	\item  Muestre que los intervalos de tiempo en el marco de referencia de la bola y el piso están relacionados por el factor gamma relevante.
	\item  Muestre que los tiempos en el marco del tren y el piso no están relacionados por el factor relevante $\gamma$. ¿Por qué pasa esto?
\end{enumerate} 


\noindent\rule{16.5cm}{0.4pt}

\textbf{Fórmulas útiles}

\begin{equation*}
\gamma = \frac{1}{\sqrt{1-\beta^2}} > 1
\end{equation*}

\begin{equation}
\beta = \frac{v}{c}
\end{equation}

\textbf{Transformaciones de Lorentz}\\
$S'$ tiene una velocidad $v$ relativa a $S$ .

\begin{align*}
x' &= \gamma(x -vt)\\
t' &= \gamma(t -vx/c^2)
\end{align*}


\begin{align*}
x &= \gamma(x' +vt')\\
t &= \gamma(t' +vx'/c^2)
\end{align*}

\textbf{Adición de velocidades}\\
$u_x$ es una velocidad medida desde $S$ y $u_x'$ desde $S'$. 

\begin{align*}
u_x &= \frac{u_x'+v}{1+vu_x'/c^2}
\end{align*}

\begin{align*}
u_x' &= \frac{u_x- v}{1-vu_x/c^2}
\end{align*}


\end{document}