\documentclass{article}
\usepackage{braket}
\usepackage[latin1]{inputenc}
\usepackage{amsfonts}
\usepackage{amsthm}
\usepackage{amsmath}
\usepackage{mathrsfs}
\usepackage{enumitem}
\usepackage[pdftex]{color,graphicx}
\usepackage{hyperref}
\usepackage{listings}
\usepackage{calligra}
\usepackage{algpseudocode} 
\DeclareFontShape{T1}{calligra}{m}{n}{<->s*[2.2]callig15}{}
\newcommand{\scripty}[1]{\ensuremath{\mathcalligra{#1}}}
\setlength{\oddsidemargin}{0cm}
\setlength{\textwidth}{490pt}
\setlength{\topmargin}{-40pt}
\addtolength{\hoffset}{-0.3cm}
\addtolength{\textheight}{4cm}
\usepackage{subcaption}
\usepackage{amssymb}
\setlength{\parindent}{0cm}

%\begin{figure}[H]
%	\centering
%	\includegraphics[scale = 0.42]{lcaoderecha}
%	\caption{Eletr\'on ligado solo al n�cleo derecho}
%	\label{fig1}
%	\end{figure}




\begin{document}
	%\tableofcontents
	%\pagebreak
\begin{center}
	\Large \textbf{C.F�sica Moderna: Taller 10}\\
	\normalsize \textbf{Introducci�n a la ecuaci�n de Schr�dinger 2}
\end{center}


	
\section{Normalizaci�n y valores esperados en 1D}

Se tiene la siguiente funci�n de onda para un part�cula:

\begin{equation*}
\psi(x,t) = A  \exp\left( -\frac{(x-x_0)^2}{4a^2} + \frac{ip_0 x}{\hbar} + i\omega_0 t    \right)
\end{equation*}	
	
\begin{enumerate}
	\item Calcule la constante de normalizaci�n $A$.
	\item Calcule el valor esperado de la posici�n.
	\item Calcule el valor esperado del momento.
\end{enumerate}	
\noindent\rule{16.5cm}{0.4pt}

\section{Valor esperado en 3D}

Una electr�n est� sujeto a la siguiente energ�a potencial:

\begin{equation}
V(\vec{r} ) = \frac{e^2}{r}
\end{equation}

Donde $ r = |\vec{r}|$. En su estado base el electr�n tiene la siguiente funci�n de onda definida sobre todo $\mathbb{R}^3$:

\begin{equation*}
\psi(\vec{r}) = \frac{1}{\sqrt{\pi a^3}} e^{-\frac{r}{a}}
\end{equation*}

Donde $a = \hbar^2/(m_e e^2)$. Calcule $\braket{V}$, el valor esperado de la energ�a potencial en el estado base.  

\noindent\rule{16.5cm}{0.4pt}

\section{Valor esperado del momento}

Una part�cula atrapada en un pozo de potencial infinito tiene la siguiente funci�n de onda:

\begin{equation}
\psi(x) =\begin{cases}
 \sqrt{\frac{2}{L}} \sin\left(\frac{2\pi x}{L}\right) & x\in[0,L]\\
 \quad 0 & x\notin[0,L]
\end{cases}
\end{equation}


\begin{enumerate}
	\item Calcule $\Delta p$.
	\item Calcule $\Delta x$.
	\item Calcule $\Delta x\Delta p$. �Se cumple el principio de incertidumbre $\Delta x\Delta p \geq \hbar/2$?.
\end{enumerate}

\noindent\rule{16.5cm}{0.4pt}

\pagebreak
\begin{center}
	\textbf{F�rmulas �tiles}
\end{center}


\end{document}