\documentclass{article}
\usepackage{braket}
\usepackage[latin1]{inputenc}
\usepackage{amsfonts}
\usepackage{amsthm}
\usepackage{amsmath}
\usepackage{mathrsfs}
\usepackage{enumitem}
\usepackage[pdftex]{color,graphicx}
\usepackage{hyperref}
\usepackage{listings}
\usepackage{calligra}
\usepackage{algpseudocode} 
\DeclareFontShape{T1}{calligra}{m}{n}{<->s*[2.2]callig15}{}
\newcommand{\scripty}[1]{\ensuremath{\mathcalligra{#1}}}
\setlength{\oddsidemargin}{0cm}
\setlength{\textwidth}{490pt}
\setlength{\topmargin}{-40pt}
\addtolength{\hoffset}{-0.3cm}
\addtolength{\textheight}{4cm}
\usepackage{subcaption}
\usepackage{amssymb}
\setlength{\parindent}{0cm}

%\begin{figure}[H]
%	\centering
%	\includegraphics[scale = 0.42]{lcaoderecha}
%	\caption{Eletr�n ligado solo al n�cleo derecho}
%	\label{fig1}
%	\end{figure}




\begin{document}
	%\tableofcontents
	%\pagebreak
\begin{center}
	\Large \textbf{C.F�sica Moderna: Taller 11}\\
	\normalsize \textbf{�tomo de Hidr�geno}
\end{center}

\section{Ecuaci�n de Schr\"{o}dinger vs modelo de Bohr}


Una de las funciones posibles para el estado $2p$ ($n=2$, $\ell=1$) del �tomo de hidr�geno es:

\begin{equation*}
\psi_{210}=\frac{Are^{-\frac{r}{2a_0}}}{a_o}\cos\theta
\end{equation*}

Calcular para ese estado:
\begin{enumerate}
	\item La constante de normalizaci�n $A$ en t�rminos de $a_0$.
	\item El valor esperado $\braket{r}$ en t�rminos de $a_0$. 
	\item La incertidumbre $\Delta r$ en la posici�n del electr�n en t�rminos de $a_0$.
		$$\Delta r=\sqrt{\langle r^2 \rangle - \langle r \rangle^2} $$
	\item De acuerdo a la teor�a  de Bohr, el valor de los radios de las �rbitas permitidas viene dada por $r_n=n^2a_0$, siendo $n$ el n�mero cu�ntico y $a_0$ el radio del estado base. �El radio para $n=2$ se encuentra dentro del intervalo establecido por el valor esperado y la incertidumbre (el intervalo $[\braket{r}-\Delta r,\braket{r}+\Delta r]$)?
\end{enumerate}

\textbf{Ayuda: }$\int_0^\infty x^n e^{-ax}~dx=n!/a^{n+1}$

\hrulefill \\

 \section{Cuantizaci�n del momento angular}
 
 
	Para un estado excitado del �tomo de hidr�geno.
	\begin{enumerate}
		\item Demuestre que el �ngulo m�nimo que puede formar el vector momento angular $\vec{L}$ con el eje $z$ es:
		\begin{eqnarray}
		(\theta_L)_{min}=\arccos\left(\frac{n-1}{\sqrt{n(n-1)}}\right)	\nonumber 
		\end{eqnarray}
		\item ?`Cu�l es la ecuaci�n correspondiente para $(\theta_L)_{max}$, el mayor �ngulo posible entre $\vec{L}$ y el eje $z$?	
	\end{enumerate}
	
%\end{multicols}


\hrulefill \\


\begin{center}
	\textbf{FORMULAS �TILES}
\end{center}


\begin{eqnarray}
&:&\text{Condici�n de Normalizaci�n:} \quad 
\int_0^{2\pi}\int_0^{\pi}\int _0^\infty\psi^*\psi r^2 dr \sin\theta d\theta d\phi = 1 \nonumber \\
&:&\text{Valores esperados:} \quad
\langle \mathcal{\hat{O}} \rangle= \int_0^{2\pi}\int_0^{\pi}\int _0^\infty \psi^* \hat{\mathcal{O}} \psi \;\underbrace{r^2 dr \sin\theta d\theta d\phi}_{dV} \nonumber \\
&:&\text{Energias Permitidas:} \quad 
E_n=-\frac{13.6}{n^2}~eV \nonumber \\
&:&\text{Momento Angular:} \quad 
L=\sqrt{\ell(\ell+1)}\hbar \quad
L_z=m_\ell\hbar \quad \cos\theta=\frac{L_z}{L}\nonumber 
\end{eqnarray}
\begin{center}
	\begin{tabular}{|c|c|c|}\hline
		N�mero Cu�ntico & S�mbolo & Posibles Valores \\ 
		\hline
		Principal & $n$ & 1,2,3,... \\
		\hline
		Orbital & $\ell$ & 0,1,2,...., n-1 \\
		\hline
		Magn�tico & $m_{\ell}$ & $0,\pm 1,\pm 2,....,\pm \ell$ \\
		\hline
	\end{tabular}
\end{center}


\end{document}