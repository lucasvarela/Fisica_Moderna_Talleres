\documentclass{article}
\usepackage{braket}
\usepackage[latin1]{inputenc}
\usepackage{amsfonts}
\usepackage{amsthm}
\usepackage{amsmath}
\usepackage{mathrsfs}
\usepackage{enumitem}
\usepackage[pdftex]{color,graphicx}
\usepackage{hyperref}
\usepackage{listings}
\usepackage{calligra}
\usepackage{algpseudocode} 
\DeclareFontShape{T1}{calligra}{m}{n}{<->s*[2.2]callig15}{}
\newcommand{\scripty}[1]{\ensuremath{\mathcalligra{#1}}}
\setlength{\oddsidemargin}{0cm}
\setlength{\textwidth}{490pt}
\setlength{\topmargin}{-40pt}
\addtolength{\hoffset}{-0.3cm}
\addtolength{\textheight}{4cm}
\usepackage{subcaption}
\usepackage{amssymb}
\setlength{\parindent}{0cm}

%\begin{figure}[H]
%	\centering
%	\includegraphics[scale = 0.42]{lcaoderecha}
%	\caption{Eletr�n ligado solo al n�cleo derecho}
%	\label{fig1}
%	\end{figure}




\begin{document}
	%\tableofcontents
	%\pagebreak
\begin{center}
	\Large \textbf{C.F�sica Moderna: Soluci�n Taller 11}\\
	\normalsize \textbf{�tomo de Hidr�geno}
\end{center}

\section{Ecuaci�n de Schr\"{o}dinger vs modelo de Bohr}


Una de las funciones posibles para el estado $2p$ ($n=2$, $\ell=1$) del �tomo de hidr�geno es:

\begin{equation*}
\psi_{210}=\frac{Are^{-\frac{r}{2a_0}}}{a_o}\cos\theta
\end{equation*}

Calcular para ese estado:
\begin{enumerate}
	\item La constante de normalizaci�n $A$ en t�rminos de $a_0$.
	\item El valor esperado $\braket{r}$ en t�rminos de $a_0$. 
	\item La incertidumbre $\Delta r$ en la posici�n del electr�n en t�rminos de $a_0$.
	$$\Delta r=\sqrt{\langle r^2 \rangle - \langle r \rangle^2} $$
	\item De acuerdo a la teor�a  de Bohr, el valor de los radios de las �rbitas permitidas viene dada por $r_n=n^2a_0$, siendo $n$ el n�mero cu�ntico y $a_0$ el radio del estado base. �El radio para $n=2$ se encuentra dentro del intervalo establecido por el valor esperado y la incertidumbre (el intervalo $[\braket{r}-\Delta r,\braket{r}+\Delta r]$)?
\end{enumerate}

\textbf{Ayuda: }$\int_0^\infty x^n e^{-ax}~dx=n!/a^{n+1}$

\begin{center}
	\textbf{\large \textbf{Soluci\'on}}	
\end{center}
Usamos la condici\'on de normalizaci\'on para calcular el valor de la constante $A$.

\begin{eqnarray}
1&=&\int_0^{2\pi}\int_0^{\pi}\int_0^\infty\Psi^*\Psi r^2 dr \sin\theta d\theta d\phi=\int_0^{2\pi}\int_0^{\pi}\int _0^\infty \left(\frac{Are^{-\frac{r}{2a_0}}}{a_0}\cos\theta\right)^2	 r^2 dr \sin\theta d\theta d\phi \nonumber \\
1&=&\frac{A^2}{a_0^2}\left[\int\limits_0^{2\pi} d\phi\int\limits_0^\pi \cos^2\theta \sin\theta d\theta \int\limits_0^\infty r^4 e^{-\frac{r}{a_0}}dr\right]=\frac{2\pi A^2}{a_0^2}\left[\int\limits_0^\pi \cos^2\theta \sin\theta d\theta \int\limits_0^\infty r^4 e^{-\frac{r}{a_0}}dr\right] \nonumber \\
1&=&\frac{2\pi A^2}{a_0^2}\left[\left.-\frac{\cos^3\theta}{3}\right|_0^\pi  \int\limits_0^\infty r^4 e^{-\frac{r}{a_0}}dr\right]=\frac{4\pi A^2}{3a_0^2}\int\limits_0^\infty r^4 e^{-\frac{r}{a_0}}dr=\frac{4\pi A^2}{3a_0^2}\left(\frac{4!}{\left(\frac{1}{a_0}\right)^5}\right) \nonumber \\
1&=&\frac{4\pi A^2}{3a_0^2}\left(24a_0^5\right) \quad \rightarrow \quad 1=32\pi A^2 a_0^3 \quad \rightarrow \quad A=\frac{1}{\sqrt{32\pi}a_0^{\frac{3}{2}}} \nonumber 
\end{eqnarray}
Por lo tanto la funci\'on de onda es:

\begin{eqnarray}
\Psi_{210}=\frac{1}{\sqrt{32\pi}a_0^{\frac{5}{2}}}re^{-\frac{r}{2a_0}}\cos\theta	\nonumber 
\end{eqnarray}


Para calcular el valor esperado de la posici\'on $\langle r \rangle$, usamos:

\begin{eqnarray}
\langle r \rangle&=&\int_0^{2\pi}\int_0^{\pi}\int_0^\infty\left(\frac{1}{\sqrt{32\pi}a_0^{\frac{5}{2}}}re^{-\frac{r}{2a_0}}\cos\theta\right)r\left(\frac{1}{\sqrt{32\pi}a_0^{\frac{5}{2}}}re^{-\frac{r}{2a_0}}\cos\theta\right) r^2 dr \sin\theta d\theta d\phi	\nonumber \\
\langle r \rangle&=&\frac{1}{32\pi a_0^{5}}\left[\int_0^{2\pi}d\phi\int_0^{\pi} \cos^2\theta \sin\theta d\theta \int_0^\infty r^5 e^{-\frac{r}{a_0}}dr \right] \nonumber \\
\langle r \rangle&=&\frac{1}{16 a_0^{5}}\left[\left.-\frac{\cos^3\theta}{3}\right|_0^\pi \left(\frac{5!}{\frac{1}{a_0^6}}\right)\right]=\frac{1}{16 a_0^{5}}\left(\frac{2}{3}\right)\left(120a_0^6\right)=5a_0 \nonumber 
\end{eqnarray}
Para calcular la incertidumbre $\Delta r$, se necesita el valor esperado de $\langle r^2 \rangle$, veamos:

\begin{eqnarray}
\langle r^2 \rangle&=&\int_0^{2\pi}\int_0^{\pi}\int_0^\infty\left(\frac{1}{\sqrt{32\pi}a_0^{\frac{5}{2}}}re^{-\frac{r}{2a_0}}\cos\theta\right)r^2\left(\frac{1}{\sqrt{32\pi}a_0^{\frac{5}{2}}}re^{-\frac{r}{2a_0}}\cos\theta\right) r^2 dr \sin\theta d\theta d\phi	\nonumber \\
\langle r^2 \rangle&=&\frac{1}{32\pi a_0^{5}}\left[\int_0^{2\pi}d\phi\int_0^{\pi} \cos^2\theta \sin\theta d\theta \int_0^\infty r^6 e^{-\frac{r}{a_0}}dr \right] \nonumber \\
\langle r^2 \rangle&=&\frac{1}{16 a_0^{5}}\left[\left.-\frac{\cos^3\theta}{3}\right|_0^\pi \left(\frac{6!}{\frac{1}{a_0^7}}\right)\right]=\frac{1}{16 a_0^{5}}\left(\frac{2}{3}\right)\left(720a_0^7\right)=30a_0^2 \nonumber \\
\Delta r&=&\sqrt{\langle r^2 \rangle-\langle r \rangle^2}=\sqrt{30a_0^2-(5a_0)^2}=\sqrt{5a_0^2}=\sqrt{5}a_0 \nonumber 
\end{eqnarray}
Usando la teor�a de Bohr, el radio de la \'orbita permitida para $n=2$ es:

\begin{eqnarray}
r_2=2^2a_0=4a_0 \nonumber 	
\end{eqnarray}
Los valores de los radios de Bohr corresponden a los valores mas probables, y deben estar dentro del rango establecido alrededor del valor esperado $\langle r\rangle \pm \Delta r$, veamos:

\begin{eqnarray}
\langle r\rangle \pm \Delta r=5a_0\pm \sqrt{5}a_0 \quad \therefore \quad \langle r\rangle \pm \Delta r=(5\pm \sqrt{5})a_0	\nonumber \\
\left[(5-\sqrt{5})a_0,(5+\sqrt{5})a_0\right]=\left[2.764a_0,7.236a_0\right] \nonumber \\
2.764a_0<4a_0<7.236a_0 \nonumber
\end{eqnarray}
Luego el valor que predice Bohr se encuentra dentro del rango establecido del valor esperado para \'este n\'umero cu\'antico.





\hrulefill \\


 \section{Cuantizaci�n del momento angular}
 
 
	Para un estado excitado del �tomo de hidr�geno.
	\begin{enumerate}
		\item Demuestre que el �ngulo m�nimo que puede formar el vector momento angular $\vec{L}$ con el eje $z$ es:
		\begin{eqnarray}
		(\theta_L)_{min}=\arccos\left(\frac{n-1}{\sqrt{n(n-1)}}\right)	\nonumber 
		\end{eqnarray}
		\item ?`Cu�l es la ecuaci�n correspondiente para $(\theta_L)_{max}$, el mayor �ngulo posible entre $\vec{L}$ y el eje $z$?	
	\end{enumerate}
	
%\end{multicols}

\begin{center}
	\textbf{\large \textbf{Soluci\'on}}		
\end{center}
El \'angulo $\theta_L$ es el que se forma entre el vector $\vec{L}$ y el eje $z$. Luego:

\begin{eqnarray}
\cos\theta_L=\frac{L_z}{L}	\quad \rightarrow \quad \theta_L=\arccos\left(\frac{L_z}{L}\right) \nonumber 
\end{eqnarray}
El \'angulo m\'as peque\~no $(\theta_L)_{min}$ se logra cuando $L_z$ tiende al valor de $L$, esto se obtiene con las siguientes condiciones:

\begin{eqnarray}
\ell=n-1 \quad &;& \quad m_\ell=\ell=n-1 \nonumber \\
L=\sqrt{\ell(\ell+1)}\hbar=\sqrt{(n-1)n} \hbar \quad &;& \quad L_z=m_\ell\hbar=(n-1)\hbar \nonumber \\
(\theta_L)_{min}=\arccos{\left(\frac{(n-1)\hbar}{\sqrt{(n-1)n} \hbar}\right)} \quad &\therefore& \quad (\theta_L)_{min}=\arccos{\left(\frac{(n-1)}{\sqrt{(n-1)n}}\right)} \nonumber \\
(\theta_L)_{min}=\arccos{\left(\sqrt{\frac{n-1}{n}}\right)}  \quad &\therefore& \quad (\theta_L)_{min}=\arccos{\left(\sqrt{1-\frac{1}{n}}\right)} \nonumber
\end{eqnarray}
Obs\'ervese que $(\theta_L)_{min}$ se aproxima a $0^\circ$ cuando $n\rightarrow \infty$. 

Para calcular el \'angulo m\'as grande $(\theta_L)_{max}$, se logra con las siguientes condiciones:

\begin{eqnarray}
\ell=n-1 \quad &;& \quad m_\ell=-\ell=-(n-1) \nonumber \\	
L=\sqrt{\ell(\ell+1)}\hbar=\sqrt{(n-1)n} \hbar \quad &;& \quad L_z=m_\ell\hbar=-(n-1)\hbar \nonumber \\
(\theta_L)_{max}=\arccos{\left(\frac{-(n-1)\hbar}{\sqrt{(n-1)n} \hbar}\right)} \quad &\therefore& \quad (\theta_L)_{max}=\arccos{\left(\frac{-(n-1)}{\sqrt{(n-1)n}}\right)} \nonumber \\
(\theta_L)_{max}=\arccos{\left(-\sqrt{\frac{n-1}{n}}\right)} \quad &\therefore&  (\theta_L)_{max}=\arccos{\left(-\sqrt{1-\frac{1}{n}}\right)} \nonumber
\end{eqnarray}
Obs\'ervese que $(\theta_L)_{max}$ se aproxima a $180^\circ$ cuando $n\rightarrow \infty$.




\hrulefill \\


\begin{center}
	\textbf{FORMULAS �TILES}
\end{center}


\begin{eqnarray}
&:&\text{Condici�n de Normalizaci�n:} \quad 
\int_0^{2\pi}\int_0^{\pi}\int _0^\infty\psi^*\psi r^2 dr \sin\theta d\theta d\phi = 1 \nonumber \\
&:&\text{Valores esperados:} \quad
\langle \mathcal{\hat{O}} \rangle= \int_0^{2\pi}\int_0^{\pi}\int _0^\infty \psi^* \hat{\mathcal{O}} \psi \;\underbrace{r^2 dr \sin\theta d\theta d\phi}_{dV} \nonumber \\
&:&\text{Energias Permitidas:} \quad 
E_n=-\frac{13.6}{n^2}~eV \nonumber \\
&:&\text{Momento Angular:} \quad 
L=\sqrt{\ell(\ell+1)}\hbar \quad
L_z=m_\ell\hbar \quad \cos\theta=\frac{L_z}{L}\nonumber 
\end{eqnarray}
\begin{center}
	\begin{tabular}{|c|c|c|}\hline
		N�mero Cu�ntico & S�mbolo & Posibles Valores \\ 
		\hline
		Principal & $n$ & 1,2,3,... \\
		\hline
		Orbital & $\ell$ & 0,1,2,...., n-1 \\
		\hline
		Magn�tico & $m_{\ell}$ & $0,\pm 1,\pm 2,....,\pm \ell$ \\
		\hline
	\end{tabular}
\end{center}


\end{document}