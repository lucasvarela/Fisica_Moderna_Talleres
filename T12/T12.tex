\documentclass{article}
\usepackage{braket}
\usepackage[latin1]{inputenc}
\usepackage{amsfonts}
\usepackage{amsthm}
\usepackage{amsmath}
\usepackage{mathrsfs}
\usepackage{enumitem}
\usepackage[pdftex]{color,graphicx}
\usepackage{hyperref}
\usepackage{listings}
\usepackage{calligra}
\usepackage{algpseudocode} 
\DeclareFontShape{T1}{calligra}{m}{n}{<->s*[2.2]callig15}{}
\newcommand{\scripty}[1]{\ensuremath{\mathcalligra{#1}}}
\setlength{\oddsidemargin}{0cm}
\setlength{\textwidth}{490pt}
\setlength{\topmargin}{-40pt}
\addtolength{\hoffset}{-0.3cm}
\addtolength{\textheight}{4cm}
\usepackage{subcaption}
\usepackage{amssymb}
\setlength{\parindent}{0cm}

%\begin{figure}[H]
%	\centering
%	\includegraphics[scale = 0.42]{lcaoderecha}
%	\caption{Eletr�n ligado solo al n�cleo derecho}
%	\label{fig1}
%	\end{figure}




\begin{document}
	%\tableofcontents
	%\pagebreak
\begin{center}
	\Large \textbf{C.F�sica Moderna: Taller 12}\\
	\normalsize \textbf{Momentum angular}
\end{center}

\section{Capa M del �tomo de hidr�geno}

Un electr�n en un �tomo de hidr�geno se encuentra en la capa M. 

\begin{enumerate}
	\item Determinar su energ�a y
	el n�mero de estados de esta configuraci�n.
	\item En cada caso calcular el �ngulo que hace el
	momemtum angular con el eje $z$ y realizar un diagrama con las direcciones permitidas.
\end{enumerate}
 


\hrulefill 

 \section{Representaci�n del operador momento angular}
 
Obtenga el operador $\hat{L}_z$ en coordenadas rectangulares y luego en coordenadas esf�ricas.
Asumiendo que el operador $\hat{L}_z$ cumple una ecuaci�n de valore propios determinar las
funciones propias correspondientes a este operador.
	
%\end{multicols}


\hrulefill 

\section{Transiciones del �tomo de hidr�geno}

Si las transiciones permitidas entre capas y subcapas en el �tomo de hidr�geno cumplen la
condici�n $\Delta \ell =\pm 1$, realizar un diagrama con los niveles y subniveles de energ�a y las
transiciones permitidas entre los diferentes estados $n$ y $\ell$.

\hrulefill 

\section{Hidr�geno en presencia de un campo magn�tico}

Un �tomo de hidr�geno se introduce en un espacio donde existe un campo magn�tico $B$ en
la direcci�n $z$. La energ�a se puede escribir de la forma $E=E_n +E_B$, donde $E_n$ es la energ�a
del �tomo en ausencia del campo magn�tico y cl�sicamente $E_B= (e/2m_e) \vec{L}\cdot\vec{B}$.

\begin{enumerate}
	\item Determine que
	sucede con los niveles de energ�a correspondientes a las subcapas s,p,d.
	\item Las reglas de
	transici�n ahora son $\Delta m_{\ell} = 0, \pm 1$; �qu� sucede con las l�neas espectrales emitidas por el
	�tomo en presencia de un campo magn�tico? Realice una representaci�n en niveles de
	energ�a.
\end{enumerate}


\hrulefill \\
\begin{center}
	\textbf{FORMULAS �TILES}
\end{center}


\begin{eqnarray}
&:&\text{Momento angular:} \quad 
\hat{L}= \hat{r}\times \hat{p} \nonumber \\
&:&\text{Coordenadas esf�ricas:} \quad 
 x=r\sin\theta \cos \phi \quad y=r\sin\theta \sin \phi \quad z= r\cos\theta \nonumber \\
&:&\text{Momento Angular:} \quad 
L=\sqrt{\ell(\ell+1)}\hbar \quad
L_z=m_\ell\hbar \quad \cos\theta=\frac{L_z}{L}\nonumber 
\end{eqnarray}
\begin{center}
	\begin{tabular}{|c|c|c|}\hline
		N�mero Cu�ntico & S�mbolo & Posibles Valores \\ 
		\hline
		Principal & $n$ & 1,2,3,... \\
		\hline
		Orbital & $\ell$ & 0,1,2,...., n-1 \\
		\hline
		Magn�tico & $m_{\ell}$ & $0,\pm 1,\pm 2,....,\pm \ell$ \\
		\hline
	\end{tabular}
\end{center}


\end{document}