\documentclass{article}
\usepackage{braket}
\usepackage[latin1]{inputenc}
\usepackage{amsfonts}
\usepackage{amsthm}
\usepackage{amsmath}
\usepackage{mathrsfs}
\usepackage{enumitem}
\usepackage[pdftex]{color,graphicx}
\usepackage{hyperref}
\usepackage{listings}
\usepackage{calligra}
\usepackage{algpseudocode} 
\DeclareFontShape{T1}{calligra}{m}{n}{<->s*[2.2]callig15}{}
\newcommand{\scripty}[1]{\ensuremath{\mathcalligra{#1}}}
\setlength{\oddsidemargin}{0cm}
\setlength{\textwidth}{490pt}
\setlength{\topmargin}{-40pt}
\addtolength{\hoffset}{-0.3cm}
\addtolength{\textheight}{4cm}
\usepackage{subcaption}
\usepackage{amssymb}
\setlength{\parindent}{0cm}
\newcommand{\e}{{\rm e}}
%\begin{figure}[H]
%	\centering
%	\includegraphics[scale = 0.42]{lcaoderecha}
%	\caption{Eletr�n ligado solo al n�cleo derecho}
%	\label{fig1}
%	\end{figure}




\begin{document}
	%\tableofcontents
	%\pagebreak
\begin{center}
	\Large \textbf{C.F�sica Moderna: Taller 13}\\
	\normalsize \textbf{Ecuaci�n de Schr\"{o}dinger en 3D}
\end{center}

\section{Pozo de potencial infinito en 3D}


Considere un pozo infinito $V(\boldsymbol{r})$ con profundidad infinita de lados $L_x ,L_y \text{ y } L_z$. 

\begin{equation*}
V(\boldsymbol{r}) = \begin{cases}
0  & \text{si } \boldsymbol{r} \in [0,L_x]\times[0,L_y]\times [0,L_z]\\
\infty & \text{de lo contrario} 
\end{cases}
\end{equation*}

El objetivo de este ejercicio es encontrar los estados estacionarios para una part�cula confinada en dicho pozo y sus energ�as permitidas. Plante� un ansatz para la soluci�n de la ecuaci�n de Schr�dinger de la forma

\begin{equation*}
\Psi(\boldsymbol{r},t) = \psi(\boldsymbol{r}) \e^{-iEt/\hbar}
\end{equation*}

donde

\begin{equation*}
\psi(\boldsymbol{r}) = X(x)Y(y)Z(z)
\end{equation*}

\begin{enumerate}
	\item Calcule $X(x),Y (y) \text{ y } Z(z)$.
	\item Usando las condiciones de frontera apropiadas encuentre las energ�as propias para el sistema.
	\item Calcular la constante de normalizaci�n para los funciones de onda de los estados estacionarios.
	\item Encuentre la condici�n que deben cumplir  $L_x ,L_y \text{ y } L_z$ para que las energ�as no sean degeneradas.
\end{enumerate}

\hrulefill 
 \section{Estado degenerados}
 
 
Una part�cula de masa $m$ se mueve en una caja tridimensional cuyas dimensiones son $L_x = L,L_y = 2L \text{ y } L_z = 2L$. 

\begin{enumerate}
	\item Encuentre	las primeras seis energ�as m�s bajas.
	\item Diga cuales de esas energ�as son degeneradas.
	\item Escriba las funciones de onda de los estados de la primera energ�a degenerada. Estas deben quedar �nicamente en t�rminos de $L$ y la posici�n. 
\end{enumerate}


\hrulefill \\
\begin{center}
	\textbf{FORMULAS �TILES}
\end{center}

Ecuaci�n de Schr�dinger independiente del tiempo:

\begin{equation*}
-\frac{\hbar^2}{2m} \nabla^2 \psi(\boldsymbol{r}) + V(\boldsymbol{r}) \psi(\boldsymbol{r}) = E \psi(\boldsymbol{r})
\end{equation*}


\end{document}