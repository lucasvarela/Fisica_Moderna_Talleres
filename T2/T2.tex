  \documentclass[12pt]{article}
 
\usepackage[margin=1in]{geometry}
\usepackage{amsmath,amsthm,amssymb}
\usepackage[spanish]{babel}
\decimalpoint
\usepackage[utf8]{inputenc}
\usepackage{enumitem, kantlipsum}
\usepackage{graphicx}
\setlength{\parindent}{0cm} 

\begin{document}
 
\begin{center}
\Large \textbf{C.Física Moderna: Taller 2}\\
\normalsize \textbf{Dilatación del tiempo y adición de velocidades}
\end{center}
 
  

\section{Piones}
 Un cúmulo de mesones $\pi^+$ (piones) se desplazan hacia un tubo de evacuación en el laboratorio Fermilab, moviéndose a $\beta= 0.92$ con respecto al laboratorio.
 	
 	\begin{enumerate}
 		\item Calcular el factor $\gamma$ para este grupo de piones.
 		\item El tiempo propio de vida media $\tau$ de los de piones es de $2.6\times1 0^{-8}~s$. Determine el valor de la vida media de los piones medidos en el laboratorio.
 		\item Si el cúmulo contenía a $50000$ piones,  ¿cuántos permanecen después de que el grupo ha viajado $50~m$ dentro del tubo? (Dato: $N=N_0e^{-t'/\tau}$, donde $t'$ es el tiempo medido en el marco del meson.)
 		\item ¿Cuál sería la respuesta al inciso anterior si hacemos caso omiso de la dilatación del tiempo?
 	\end{enumerate}
\noindent\rule{16.5cm}{0.4pt}

\section{Gemelos}

Luisa aborda una nave espacial y se aleja de la Tierra a una velocidad constante de $0.45c$ hacia una galaxia muy lejana. Un a\~no m\'as tarde seg\'un los relojes de la Tierra, su gemelo Sebastián, aborda una segunda nave espacial y la sigue a una velocidad constante de $0.95c$ en la misma dirección. 
\begin{enumerate}
	\item Cuando Sebastián alcanza a Luisa, ¿Cuál será la diferencia de edad?
	\item ¿Qué gemelo será mayor?
\end{enumerate}


\noindent\rule{16.5cm}{0.4pt}

\section{Naves espaciales}

Usted observa que dos naves espaciales viajan en dirección opuesta con una rapidez de $0.9 c$. Determine la rapidez relativa entre las naves.\\

\begin{enumerate}
	\item Determine la rapidez relativa entre las naves.
	\item Repita el calculo para el inciso anterior sin tener en cuenta la relatividad especial. Diga por qué la respuesta que obtiene no es aceptable para la teoría de la relatividad especial. 
\end{enumerate}
\noindent\rule{16.5cm}{0.4pt}
\pagebreak

\textbf{Fórmulas útiles}

\begin{equation*}
\gamma = \frac{1}{\sqrt{1-(v/c)^2}} > 1
\end{equation*}
 
\begin{equation*}
\beta = \frac{v}{c}
\end{equation*}

 

\textbf{Transformaciones de Lorentz}\\



\begin{align*}
x' &= \gamma\left(x - vt\right)\\
t' &= \gamma\left(t - xv/c^2\right)
\end{align*}

\textbf{Adición de velocidades}\\
$u_x$ es una velocidad medida desde $S$ y $u_x'$ desde $S'$. 

\begin{align*}
u_x &= \frac{u_x'+v}{1+vu_x'/c^2}
\end{align*}

\begin{align*}
u_x' &= \frac{u_x- v}{1-vu_x/c^2}
\end{align*}


\end{document}