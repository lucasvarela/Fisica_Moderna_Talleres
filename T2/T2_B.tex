  \documentclass[12pt]{article}
 
\usepackage[margin=1in]{geometry}
\usepackage{amsmath,amsthm,amssymb}
\usepackage[spanish]{babel}
\decimalpoint
\usepackage[utf8]{inputenc}
\usepackage{enumitem, kantlipsum}
\usepackage{graphicx}
\setlength{\parindent}{0cm} 

\begin{document}
 
\begin{center}
\Large \textbf{C.Física Moderna: Taller 2 B}
\end{center}
 
  

\section{Partícula}
 
 
 Una partícula moviéndose a 0.8c en el marco de referencia de
 laboratorio decae después de viajar 3m.
 
 \begin{enumerate}
 	\item ¿Cuanto tiempo existió esta partícula para un observador en el laboratorio?
 	\item ¿Cuanto tiempo existió esta partícula para un observador en un marco de referencia en reposo respecto a la partícula?
 \end{enumerate}
 
\noindent\rule{16.5cm}{0.4pt}

\section{Ángulo}

Una vara de un metro(longitud propia) forma un ángulo de 30$^{\circ}$ con respecto al eje $x'$ de $S'$. 
Un observador en un marco $S$ ve que la vara forma un ángulo de 45$^{\circ}$ respecto al eje $x$. El marco $S'$ se mueve con una rapidez $v$ en la dirección $x$ respecto a $S$.

\begin{enumerate}
	\item Determine la rapidez $v$. 
	\item Encuentre la longitud de la vara vista desde $S$.
\end{enumerate}


\noindent\rule{16.5cm}{0.4pt}


\pagebreak

\textbf{Fórmulas útiles}

\begin{equation*}
\gamma = \frac{1}{\sqrt{1-(v/c)^2}} > 1
\end{equation*}
 
\begin{equation*}
\beta = \frac{v}{c}
\end{equation*}

 

\textbf{Transformaciones de Lorentz}\\



\begin{align*}
x' &= \gamma\left(x - vt\right)\\
t' &= \gamma\left(t - xv/c^2\right)
\end{align*}

\textbf{Adición de velocidades}\\
$u_x$ es una velocidad medida desde $S$ y $u_x'$ desde $S'$. 

\begin{align*}
u_x &= \frac{u_x'+v}{1+vu_x'/c^2}
\end{align*}

\begin{align*}
u_x' &= \frac{u_x- v}{1-vu_x/c^2}
\end{align*}


\end{document}