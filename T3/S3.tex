  \documentclass[12pt]{article}
 
\usepackage[margin=1in]{geometry}
\usepackage{amsmath,amsthm,amssymb}
\usepackage[spanish]{babel}
\decimalpoint
\usepackage[utf8]{inputenc}
\usepackage{enumitem, kantlipsum}
\usepackage{graphicx}
\setlength{\parindent}{0cm} 

\begin{document}
 
\begin{center}
\Large \textbf{C.Física Moderna: Taller 3}\\
\normalsize \textbf{Solución}
\end{center}
 
  

\section{Efecto Doppler}

Desde una galaxia A se emite luz a $550\; nm$(verde). La galaxia A tiene una velocidad relativa a las galaxias B y C de $v_B$ y $v_C$ respectivamente. Sabiendo que las galaxias ven lineas de absorción  de $450\; nm$(azul) y $700\;nm$(rojo) respectivamente, calcule $v_B$ y $v_C$. Diga si se están acercando o alejando a la galaxia A.\\
\noindent\rule{16.5cm}{0.4pt}
\begin{center}
	\textbf{Solución}	
\end{center}


Utilizando la fórmula para efecto doppler relativista se tiene:

\begin{equation*}
 \frac{\lambda_o}{\lambda_s} = \frac{\sqrt{1+\beta}}{\sqrt{1-\beta}}
\end{equation*}

Despejamos para la velocidad:

\begin{equation}
\beta = \frac{\left( \frac{\lambda_o}{\lambda_s}\right)^2-1}{1+\left( \frac{\lambda_o}{\lambda_s}\right)^2} =\frac{\lambda_o^2-\lambda_s^2}{\lambda_s^2+\lambda_o^2}
\end{equation}


Por lo tanto se tiene:


\begin{align*}
\beta_B = \frac{ 450^2-550^2}{550^2 + 450^2} =- \frac{20}{101} = -0.198 \\
\beta_C = \frac{700^2-550^2}{550^2 + 700^2} = \frac{75}{317} = 0.237
\end{align*}

Por lo tanto la galaxia $B$ se se acerca a $A$ mientras que $C$ se aleja de ella. Note que esto se podía saber sin hacer ningún cálculo: Si la fuente se acerca al observador, la frecuencia con la que llega aumenta y por lo tanto la longitud de onda disminuye(ya que $\lambda = c/f$). 

\noindent\rule{16.5cm}{0.4pt}


\section{Energía relativa}

Dos partículas de masa en reposo $m_0$ se dirigen al mismo punto con velocidades $v_1 = -v_2$ relativas
a un marco de referencia inercial. ¿Cuál es la energía total de una partícula medida desde el marco
en reposo de la otra?\\
\noindent\rule{16.5cm}{0.4pt}
\begin{center}
	\textbf{Solución}	
\end{center}

La velocidad de una partícula vista por la otra es:


\begin{equation}
v'= \frac{2v}{1+ (v/c)^2}
\end{equation}

Por lo que la energía es:

\begin{equation}
E' = m_0 c^2  \gamma(v') =  m_0 c^2 \frac{1+ (v/c)^2}{1- (v/c)^2}
\end{equation}


\noindent\rule{16.5cm}{0.4pt}



\section{Colisión totalmente inelástica}


 Una partícula de masa $m$ y rapidez $v$ colisiona y se queda pegada a una partícula estacionaria
de masa $M$. ¿Cuál es la velocidad final de la partícula compuesta?\\
\noindent\rule{16.5cm}{0.4pt}
\begin{center}
	\textbf{Solución}	
\end{center}

El momento y energía inicial son:

\begin{align*}
p_i &= mv\gamma(v)\\
E_i & = mc^2\gamma(v) + Mc^2
\end{align*}

Luego de la colisión el momento y energía final son:

\begin{align*}
p_f &= M_f v_f\gamma(v_f)\\
E_f & = M_fc^2\gamma(v_f) 
\end{align*}

Utilizando la conservación de momento y energía:

\begin{align*}
 mv\gamma(v) &= M_f v_f\gamma(v_f)\\
 mc^2\gamma(v) + Mc^2& = M_fc^2\gamma(v_f) 
\end{align*}

Dividiendo las últimas dos igualdades:

\begin{equation}
\frac{ mv\gamma(v)}{mc^2\gamma(v) + Mc^2} = \frac{v_f}{c^2}
\end{equation}


Por lo que la velocidad final es:

\begin{equation}
v_f = v \frac{m\gamma(v)}{m\gamma(v) + M}
\end{equation}
\noindent\rule{16.5cm}{0.4pt}


\textbf{Fórmulas útiles}

Indice $o$ es para observador y $s$ para fuente(source).
\begin{equation*}
\frac{f_s}{f_o} = \frac{\lambda_o}{\lambda_s} = \frac{\sqrt{1+\beta}}{\sqrt{1-\beta}}
\end{equation*}


\textbf{Adición de velocidades}\\
$u_x$ es una velocidad medida desde $S$ y $u_x'$ desde $S'$. 

\begin{align*}
u_x &= \frac{u_x'+v}{1+vu_x'/c^2}
\end{align*}

\textbf{Momento y energía relativista}\\

Para las siguientes formulas $m$ es la masa en reposo.

\begin{align*}
p &= m  v \gamma\\
E &= m c^2\gamma\\
K &=  m  c^2 (\gamma-1)\\
E^2 &= (pc)^2 + (mc^2)^2
\end{align*}


\end{document}