  \documentclass[12pt]{article}
 
\usepackage[margin=1in]{geometry}
\usepackage{amsmath,amsthm,amssymb}
\usepackage[spanish]{babel}
\decimalpoint
\usepackage[utf8]{inputenc}
\usepackage{enumitem, kantlipsum}
\usepackage{graphicx}
\setlength{\parindent}{0cm} 

\begin{document}
 
\begin{center}
\Large \textbf{C.Física Moderna: Taller 3}\\
\normalsize \textbf{Efecto doppler y mecánica relativista}
\end{center}
 
  

\section{Efecto Doppler}

Desde una galaxia A se emite luz a $550\; nm$(verde). La galaxia A tiene una velocidad relativa a las galaxias B y C de $v_B$ y $v_C$ respectivamente. Sabiendo que las galaxias ven lineas de absorción  de $450\; nm$(azul) y $700\;nm$(rojo) respectivamente, calcule $v_B$ y $v_C$. Diga si se están acercando o alejando a la galaxia A.





\noindent\rule{16.5cm}{0.4pt}



\section{Energía relativa}

Dos partículas de masa en reposo $m_0$ se dirigen al mismo punto con velocidades $v_1 = -v_2$ relativas
a un marco de referencia inercial. ¿Cuál es la energía total de una partícula medida desde el marco
en reposo de la otra?





\noindent\rule{16.5cm}{0.4pt}



\section{Colisión totalmente inelástica}


 Una partícula de masa $m$ y rapidez $v$ colisiona y se queda pegada a una partícula estacionaria
de masa $M$. ¿Cuál es la velocidad final de la partícula compuesta?

\noindent\rule{16.5cm}{0.4pt}


\textbf{Fórmulas útiles}

Indice $o$ es para observador y $s$ para fuente(source).
\begin{equation*}
\frac{f_s}{f_o} = \frac{\lambda_o}{\lambda_s} = \frac{\sqrt{1+\beta}}{\sqrt{1-\beta}}
\end{equation*}


\textbf{Adición de velocidades}\\
$u_x$ es una velocidad medida desde $S$ y $u_x'$ desde $S'$. 

\begin{align*}
u_x &= \frac{u_x'+v}{1+vu_x'/c^2}
\end{align*}

\textbf{Momento y energía relativista}\\

Para las siguientes formulas $m$ es la masa en reposo.

\begin{align*}
p &= m  v \gamma\\
E &= m c^2\gamma\\
K &=  m  c^2 (\gamma-1)\\
E^2 &= (pc)^2 + (mc^2)^2
\end{align*}


\end{document}