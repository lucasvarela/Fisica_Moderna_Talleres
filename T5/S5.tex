  \documentclass[12pt]{article}
 
\usepackage[margin=1in]{geometry}
\usepackage{amsmath,amsthm,amssymb}
\usepackage[spanish]{babel}
\decimalpoint
\usepackage[utf8]{inputenc}
\usepackage{enumitem, kantlipsum}
\usepackage{graphicx}
\newcommand{\angstrom}{\mbox{\normalfont\AA}}
\setlength{\parindent}{0cm} 

\begin{document}
 
\begin{center}
\Large \textbf{C.Física Moderna: Solución Taller 5}\\
\normalsize \textbf{Radiación de cuerpo negro y función de trabajo}
\end{center}
 
  

\section{Radiación de cuerpo negro}


Una bombilla incandescente de 40 W irradia debido a un filamento de tungsteno operando
a 3300 K. Asumiendo que la bombilla irradia como un cuerpo negro, responda las siguientes preguntas:
\begin{enumerate}
	\item  ¿Cuáles son la frecuencia $\nu_{\text{max}}$  y la longitud de onda máxima $\lambda_{\text{max}}$  en el máximo de la distribución espectral $u(\lambda,T)$?*
	\item  Si suponemos que $\nu_{\text{max}}$ es una buena aproximación de la frecuencia promedio de los fotones
	emitidos por la bombilla, ¿cuántos fotones está radiando la bombilla por segundo?
	\item  ¿Si usted está observando la bombilla a $5$ m de distancia, cuántos fotones entran a su ojo
	por segundo? (El diámetro de su pupila es aproximadamente de 5.0 mm.)
\end{enumerate}


\noindent\rule{16.5cm}{0.4pt}


\begin{center}
	\textbf{Solución}
\end{center}


1.1 Usando la ley de Wien se obtiene la longitud de onda $\lambda_{\text{max}} = 878 nm$. La frecuencia es entonces $\nu = c/\lambda$. Por lo que  $\nu_{\text{max}} = 3.42 \times 10^{14}\text{Hz}$.

1.2 La energía promedio por fotón es:

\begin{equation*}
E =  h f = 2.27 \times 10^{-19} \text{J}
\end{equation*}

Dividimos la potencia entre la energía promedio para obtener el número promedio de fotones por segundo:

\begin{equation*}
\frac{P}{E} = 1.76 \times 10^{20} \text{s}^{-1}
\end{equation*}

1.3 La bombilla irradia uniformemente. Por lo que queremos saber que proporción de los fotones emitidos cae sobre el ojo. Para ello se calcula la razón entre el área del ojo y la esfera de radio 5m:

\begin{align*}
\frac{A_{\text{ojo}}}{A_{5\text{cm}}} &= \frac{\pi r^2}{4 \pi R^2} \\
&= \frac{1}{4} \left(\frac{r}{R}\right)^2\\
&= \frac{1}{4} \left(\frac{2.5 \times 10^{-3}}{5}\right)^2\\
&= \frac{10^{-6}}{16}
\end{align*}

Y con esto la cantidad de fotones que llegan al ojo es:

\begin{equation*}
N = \frac{A_{\text{ojo}}}{A_{5\text{cm}}} \frac{P}{E} = 1.1 \times 10^{13} \text{s}^{-1}
\end{equation*}

\noindent\rule{16.5cm}{0.4pt}


\section{Función de trabajo}

 Una superficie de potasio se ilumina con luz ultravioleta de longitud de onda $\lambda = 2500  \angstrom $.
Si la función trabajo del potasio es $2.21$ eV, calcular la máxima rapidez que logran los
electrones emitidos.


\noindent\rule{16.5cm}{0.4pt}


\begin{center}
	\textbf{Solución}
\end{center}

La ecuación relevante es:

\begin{equation*}
K_{\text{max}} = \frac{h c}{\lambda} - \phi
\end{equation*}

Al calcular se obtiene $K_{\text{max}} = 2.75 $ eV. Ahora usando la definición de energía cinética, obtenemos la siguiente expresión para la velocidad:


\begin{align*}
v_{\text{max}} &= \sqrt{\frac{2K_{\text{max}}}{m}}\\
&= 9.84 \times 10^5 \frac{\text{m}}{\text{s}}
\end{align*}



\noindent\rule{16.5cm}{0.4pt}


\section{Efecto fotoeléctrico}

El emisor de un tubo fotoeléctrico tiene una longitud de onda umbral $\lambda_0= 6000 \angstrom$. Calcular la longitud de onda de la luz incidente sobre el tubo si el potencial de frenado para esta luz es 2.5 V.


\begin{center}
	\textbf{Solución}
\end{center}

De acuerdo a la teoría del efecto fotoeléctrico. La longitud de onda umbral me permite
calcular la función trabajo del material:

\begin{equation*}
K_{\text{max}} = \frac{hc}{\lambda} - \phi_0
\end{equation*}

Si $K_{\text{max}} = 0$, entonces los fotoelectrones apenas se desprenden de la superficie del metal
y $\lambda = \lambda_0$ (Longitud de onda Umbral). Por lo tanto podemos calcular la función trabajo
mediante:

\begin{equation*}
\phi_0 = \frac{hc}{\lambda_0} = 2.068 eV
\end{equation*}


Dado que el potencial de frenado es una medida de la energía cinética máxima de los foto-
electrones, tenemos que $K_{\text{max}} = e\Delta V$ (donde $e$ es la carga de un electrón). Con esto tenemos la ecuación para la longitud de onda dada por:


\begin{equation*}
 e\Delta V = \frac{hc}{\lambda}- \phi_0
\end{equation*}

Despejando se obtiene:

\begin{equation*}
\lambda = \frac{hc}{e\Delta V +\phi_0} = 2716.3  \angstrom
\end{equation*}


\noindent\rule{16.5cm}{0.4pt}



\begin{center}
\textbf{Fórmulas útiles}
\end{center}




Ley de Wien:



\begin{align*}
\lambda_{\text{max}} T = 2.9 \times 10^{-3} \text{m} \cdot \text{K}
\end{align*}


Energía de un fotón con frecuencia $\nu$:


\begin{equation*}
E  = h \nu 
\end{equation*}


Función de trabajo

\begin{equation*}
K_{\text{max}} = h \nu - \phi
\end{equation*}


*Note que si le pidieran la frecuencia para la cuál la distribución de frecuencias $u(\nu,T)$ tiene su máximo, la respuesta cambiaría.
\end{document}