  \documentclass[12pt]{article}
 
\usepackage[margin=1in]{geometry}
\usepackage{amsmath,amsthm,amssymb}
\usepackage[spanish]{babel}
\decimalpoint
\usepackage[utf8]{inputenc}
\usepackage{enumitem, kantlipsum}
\usepackage{graphicx}
\newcommand{\angstrom}{\mbox{\normalfont\AA}}
\setlength{\parindent}{0cm} 

\begin{document}
 
\begin{center}
\Large \textbf{C.Física Moderna: Taller 5}\\
\normalsize \textbf{Radiación de cuerpo negro y función de trabajo}
\end{center}
 
  

\section{Radiación de cuerpo negro}


Una bombilla incandescente de 40 W irradia debido a un filamento de tungsteno operando
a 3300 K. Asumiendo que la bombilla irradia como un cuerpo negro, responda las siguientes preguntas:
\begin{enumerate}
	\item  ¿Cuáles son la frecuencia $\nu_{\text{max}}$  y la longitud de onda máxima $\lambda_{\text{max}}$  en el máximo de la distribución espectral $u(\lambda,T)$?*
	\item  Si suponemos que $\nu_{\text{max}}$ es una buena aproximación de la frecuencia promedio de los fotones
	emitidos por la bombilla, ¿cuántos fotones está radiando la bombilla por segundo?
	\item  ¿Si usted está observando la bombilla a $5$ m de distancia, cuántos fotones entran a su ojo
	por segundo? (El diámetro de su pupila es aproximadamente de 5.0 mm.)
\end{enumerate}


\noindent\rule{16.5cm}{0.4pt}

\section{Función de trabajo}

 Una superficie de potasio se ilumina con luz ultravioleta de longitud de onda $\lambda = 2500\angstrom$.
Si la función trabajo del potasio es $2.21$ eV, calcular la máxima rapidez que logran los
electrones emitidos.






\noindent\rule{16.5cm}{0.4pt}



\section{Efecto fotoeléctrico}

El emisor de un tubo fotoeléctrico tiene una longitud de onda umbral $\lambda_0= 6000\angstrom$. Calcular la longitud de onda de la luz incidente sobre el tubo si el potencial de frenado para esta luz es 2.5 V.





\noindent\rule{16.5cm}{0.4pt}





\begin{center}
\textbf{Fórmulas útiles}
\end{center}




Ley de Wien:



\begin{align*}
\lambda_{\text{max}} T = 2.9 \times 10^{-3} \text{m} \cdot \text{K}
\end{align*}


Energía de un fotón con frecuencia $\nu$:


\begin{equation*}
E  = h \nu 
\end{equation*}


Función de trabajo

\begin{equation*}
K_{\text{max}} = h \nu - \phi
\end{equation*}


*Note que si le pidieran la frecuencia para la cuál la distribución de frecuencias $u(\nu,T{\tiny })$ tiene su máximo, la respuesta cambiaría.
\end{document}