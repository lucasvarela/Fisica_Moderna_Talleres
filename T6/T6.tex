  \documentclass[12pt]{article}
 
\usepackage[margin=1in]{geometry}
\usepackage{amsmath,amsthm,amssymb}
\usepackage[spanish]{babel}
\decimalpoint
\usepackage[utf8]{inputenc}
\usepackage{enumitem, kantlipsum}
\usepackage{graphicx}
\newcommand{\angstrom}{\mbox{\normalfont\AA}}
\setlength{\parindent}{0cm} 

\begin{document}
 
\begin{center}
\Large \textbf{C.Física Moderna: Taller 6}\\
\normalsize \textbf{Ley de Bragg, efecto Compton, efecto fotoeléctrico y ley de Stefan Boltzmann }
\end{center}
 
  

\section{Efecto Compton: Teoría }

Describa el efecto Compton. Precisamente el escenario inicial y final. Luego
escriba las condiciones de conservación de energía y momento para el
experimento de Compton.


\noindent\rule{16.5cm}{0.4pt}





\section{Efecto Compton}

	En un experimento particular de efecto Compton se encuentra que la
	longitud de onda incidente $\lambda_1$ cambia en un $1.5\%$ cuando el ángulo de
	desplazamiento es $\theta = 120^{\circ}$.
	
	\begin{enumerate}
		\item  ¿Cuál es el valor de $\lambda_1$?
		\item   ¿Cuál será la longitud
		de onda del fotón dispersado cuando el ángulo sea de $75^{\circ}$? Asuma que la luz que incide tiene la longitud de onda calculada en el inciso anterior.
	\end{enumerate}

\noindent\rule{16.5cm}{0.4pt}


\section{Distancia interplanar de una muestra cristalina}

	En un experimento de difracción de rayos X ($\lambda= 0.1542$ nm) estos inciden
	sobre una muestra cristalina y como resultado el ángulo de dispersión primario
	es de $\theta = 19,3^{\circ}$. Determinar la distancia interplanar de la muestra cristalina.

\noindent\rule{16.5cm}{0.4pt}

\section{Difracción de segundo orden}

	Un cristal difractan rayos X. El espectro de primer orden corresponde a un
	ángulo de $6.5^{\circ}$ y la distancia entre planos es de $2.81 \times 10^{-10}$ m. Determinar la
	longitud de onda de los rayos X y la posición del espectro de segundo orden.

\noindent\rule{16.5cm}{0.4pt}


\section{Ley de Stefan Boltzmann}

	Para la radiación de cuerpo negro se tiene una radiancia espectral dada por:
	
	\begin{equation}
	u(\lambda,T) = \frac{8\pi h c \lambda^{-5}}{e^{\frac{hc}{\lambda k T}}-1}
	\end{equation}
	
	
	\begin{itemize}
		\item Muestre que la densidad de energía total en una cavidad de cuerpo negro es proporcional a
		$T^4$.
		De una expresión explicita para la constante de proporcionalidad $\alpha$.
		\begin{equation*}
		U(T) = \alpha T^4
		\end{equation*}
		\item Utilice la densidad de energía irradiada para obtener la ley de Stefan Boltzmann:
		$$I = \sigma T^4$$
		De la expresión para $\sigma$(déjela expresada, no encuentre el valor numérico). 
	\end{itemize}

\noindent\rule{16.5cm}{0.4pt}
\\

\begin{center}
\textbf{Fórmulas útiles}
\end{center}



Efecto Compton:

\begin{equation*}
\lambda_f - \lambda_i = \frac{h}{mc}(1-\cos\theta)
\end{equation*}

Ley de Bragg
\begin{equation*}
n \lambda = 2 d \sin\theta
\end{equation*}
Energía de un fotón con frecuencia $\nu$:


\begin{equation*}
E  = h \nu 
\end{equation*}


Función de trabajo

\begin{equation*}
K_{\text{max}} = h \nu - \phi
\end{equation*}

Integral útil $ \int_{0}^{\infty}\frac{x^3}{e^{x}-1}= \frac{\pi^4}{15}$.
\end{document}