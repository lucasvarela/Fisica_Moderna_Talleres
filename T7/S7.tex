  \documentclass[12pt]{article}
 
\usepackage[margin=1in]{geometry}
\usepackage{amsmath,amsthm,amssymb}
\usepackage[spanish]{babel}
\decimalpoint
\usepackage[utf8]{inputenc}
\usepackage{enumitem, kantlipsum}
\usepackage{graphicx}
\newcommand{\angstrom}{\mbox{\normalfont\AA}}
\setlength{\parindent}{0cm} 
\usepackage{color}

\begin{document}
 
\begin{center}
\Large \textbf{C.Física Moderna: Taller 7}\\
\normalsize \textbf{Espectros atómicos y modelo de Bohr}
\end{center}
 
  

\section{Transiciones del átomo de hidrógeno}

Se tienen las siguientes cuatro transiciones del átomo de hidrógeno:

\begin{enumerate}[label=\alph*)]
	\item $n_i= 2\quad n_f= 5$
	\item $n_i=5 \quad n_f= 3$
	\item $n_i=7 \quad n_f=4$
	\item $n_i=4 \quad n_f=7$
\end{enumerate}

\begin{enumerate}
	\item  ¿Para cuál transición se emite los fotones con menor longitud de onda? 
	\item ¿Para cuál transición gana mayor
	energía el átomo?
	\item  ¿Para cuáles transiciones perdió energía el átomo?
\end{enumerate}



\begin{center}
	\textbf{Solución}
\end{center}

Primero identificamos para cada transición que paso. Cuando se pasa de un nivel de energía mayor a uno menor ($n_f>n_i$), el átomo pierde energía. Por lo tanto tenemos:

\begin{enumerate}[label=\alph*)]
	\item $n_i= 2\quad n_f= 5\qquad$  Absorción 
	\item $n_i=5 \quad n_f= 3 \qquad$ \color{yellow} Emisión \color{black}
	\item $n_i=7 \quad n_f=4\qquad$  \color{yellow} Emisión \color{black}
	\item $n_i=4 \quad n_f=7\qquad$	Absorción 
\end{enumerate}
1.1 Para las transiciones donde el átomo emite luz se tiene perdida de energía. Es decir las transiciones (b) y (c).\\

1.2 Dado que estamos hablando de emisión, solo debemos comparar los procesos (b) y (c). Al calcular con la formula de Rydberg se obtiene:

\begin{align*}
\frac{1}{\lambda_{5\rightarrow 3}} = R\left(\frac{1}{3^2}-\frac{1}{5^2}\right) = R\frac{16}{225}\approx 0.071 R\\
\frac{1}{\lambda_{7\rightarrow 4}} = R\left(\frac{1}{4^2}-\frac{1}{7^2}\right) = R\frac{33}{784}\approx 0.042R 
\end{align*}

Es claro que $\lambda_{5\rightarrow 3}$ tiene menor longitud de onda ya que su inverso multiplicativo es mayor. Por lo tanto el proceso (b) tiene la transición que emite el fotón con menor longitud de onda.\\

1.3 Si el átomo gana energía tenemos que estar hablando de la transición (a) o (d). Nuevamente debemos calcular la longitud de onda del fotón, solo que en este caso sera absorbido. Por lo tanto tomamos el valor absoluto para la serie de Rydberg:

\begin{align*}
\frac{1}{\lambda_{2\rightarrow 5}} = R\left(\frac{1}{2^2}-\frac{1}{5^2}\right) = R\frac{21}{100}&\approx 0.21 R\\
\frac{1}{\lambda_{4\rightarrow 7}} = R\left(\frac{1}{4^2}-\frac{1}{7^2}\right) = R\frac{33}{784}&\approx 0.042R 
\end{align*}
 
La transición para la cual gana mayor energía es la que tenga un mayor inverso multiplicativo, esto se puede ver con la energía de un fotón que es proporcional al inverso multiplicativo de la longitud de onda. Entonces la transición en la que se gana mayor energía es la (d).



\noindent\rule{16.5cm}{0.4pt}





\section{Serie de Lyman para el hidrógeno}

Calcule la longitud más grande y pequeña en la serie de Lyman para el hidrógeno. Diga la transición electrónica
que da lugar a cada una. ¿Hay lineas espectrales de Lyman que están en el espectro visible?.
	

\begin{center}
	\textbf{Solución}
\end{center}

La serie de Lyman es la formula de Rydberg con $n_f = 1$:

\begin{equation*}
\frac{1}{\lambda} = R\left(1-\frac{1}{n_i^2}\right)
\end{equation*}

Vemos que los valores del máximo y mínimo de longitud de onda para esta serie se obtienen en $n_i =2$ y $n_i = \infty$
respectivamente.

\begin{align*}
\lambda_{\text{min}} &= 91.175 \text{ nm} \qquad (n_i= \infty)\\
\lambda_{\text{max}} &= 121.57 \text{ nm} \qquad (n_i= 2)	
\end{align*}

Se observa que todas las emisiones son en la banda ultravioleta, por lo que no hay ninguna línea espectral en el
espectro visible (390-700 nm).\\
\noindent\rule{16.5cm}{0.4pt}



\section{Positronio}

El Positronio es un átomo similar al de hidrógeno que consiste de un positrón y un electrón dando vueltas el uno
alrededor del otro. Usando el modelo de Bohr:

\begin{enumerate}
	\item Encuentre los radios permitidos (relativos al centro de masa) $a_{n,\text{Ps}}$ y las
	energías permitidas del sistema $E_{n,\text{Ps}}$.
	\item Calcule la razón de la energía enésima del Positronio a la del átomo de hidrógeno ($E_{n,H}/E_{n,\text{Ps}}$).
\end{enumerate}

\textbf{Ayuda: El radio de Bohr y la mínima energía  para el átomo de hidrógeno son:} $a_{o,H} = 0.53 \angstrom$ y  $E_{1,H} = -13.6 \text{ eV}$.


\begin{center}
	\textbf{Solución}
\end{center}

Se usara el subindice $\text{Ps}$ para el positronio y $H$ para el hidrógeno. El radio de la orbita fundamental(de mínima energía) para el positronio es:

\begin{equation}
a_{o,\text{Ps}} = \frac{\hbar}{\mu_{\text{Ps}} c \alpha}
\end{equation}

Para calcularla necesitamos la masa reducida del sistema:

\begin{equation}
\mu_{\text{Ps}} = \left(\frac{1}{m_e}+\frac{1}{m_e} \right)^{-1} = \frac{m_e}{2}
\end{equation}

Donde se usó que la masa del positrón es la misma que la del electrón. Ahora se repite el mismo procedimiento para el hidrógeno:

\begin{equation}
a_{o,H} = \frac{\hbar}{\mu_{H} c \alpha}
\end{equation}


\begin{equation}
\mu_{H} = \left(\frac{1}{m_p}+\frac{1}{m_e} \right)^{-1} = \frac{m_e m_p}{m_e+ m_p} \approx \frac{m_e m_p}{m_p} = m_e 
\end{equation}

Donde he usado que la masa del protón es muchísimo más grande que la del electrón. Reuniendo estos resultados tenemos:

\begin{align*}
a_{o,\text{Ps}} &= \frac{2\hbar}{m_e c \alpha}\\
a_{o,H} &= \frac{\hbar}{m_e  c \alpha}
\end{align*}

El radio mínimo del positronio es dos veces el del hidrógeno:

\begin{align*}
a_{o,\text{Ps}} = 2 a_{o,H} 
\end{align*}

Esto es útil ya que el radio del hidrógeno es conocido $a_{o,H} = 0.53 \angstrom$, por lo que:

\begin{align*}
a_{o,\text{Ps}} = 1.06 \angstrom
\end{align*}

Ahora de acuerdo al modelo de Bohr para un átomo parecido al hidrógeno tenemos que los radios permitidos $a_{n,\text{Ps}}$ están dados por:

\begin{equation}
a_{n,\text{Ps}} = a_{o,\text{Ps}} n^2 = n^2  1.06 \angstrom
\end{equation}

La energía para un átomo hidrogenoide en el modelo de Bohr está dada por:

\begin{equation}
E_n = -\frac{(k_e e^2)^2 \mu }{2 \hbar^2 n^2}
\end{equation}

Reemplazando en la expresión anterior la masa reducida del Positronio obtenemos su energía enésima. Una forma de hacerlo sin usar la calculadora es utilizando nuevamente la energía del hidrógeno como referencia. Tenemos que la energía del Positron es:

\begin{equation*}
E_{n,\text{Ps}} = -\frac{(k_e e^2)^2 m_e/2 }{2 \hbar^2 n^2}  =  \frac{1}{2} \left(-\frac{(k_e e^2)^2 m_e }{2 \hbar^2 n^2}\right) =  \frac{1}{2}E_{n,H} 
\end{equation*}

Es decir

\begin{equation*}
E_{n,\text{Ps}} =\frac{1}{2}E_{n,H} 
\end{equation*}

Las energías del hidrógeno las conocemos:



\begin{equation*}
E_{n,H} =- \frac{13.6}{n^2}\text{eV}
\end{equation*}

De donde obtenemos las del Positronio:

\begin{equation*}
E_{n,\text{Ps}} =  - \frac{6.8 }{n^2} \text{eV}
\end{equation*}


Y en el proceso ya obtuvimos el último resultado:


\begin{equation*}
\frac{E_{n,H}}{E_{n,\text{Ps}}}  =2
\end{equation*}

\noindent\rule{16.5cm}{0.4pt}

\section{Energía para una transición}

Calcule la energía de un foton que podría causar una transición electrónica de:


\begin{enumerate}[label=\alph*)]
	\item $n_i = 4$ a $n_f = 5$ 
	\item $n_i = 5$ a $n_f = 6$
\end{enumerate}


\begin{center}
	\textbf{Solución}
\end{center}

4.1 La serie de Rydberg nos da la longitud de onda:

\begin{equation*}
\frac{1}{\lambda_{4\rightarrow 5}} = R\left(\frac{1}{4^2}-\frac{1}{5^2}\right) = \frac{9}{400} R
\end{equation*}

La energía de un fotón con esa longitud de onda es:

\begin{equation*}
E = \frac{hc}{\lambda_{4\rightarrow 5}}  =4.9 \times 10^{-20} \text{ J} = 0.31  \text{ eV}
\end{equation*}



4.2 La longitud de onda es:

\begin{equation*}
\frac{1}{\lambda_{5\rightarrow 6}} = R\left(\frac{1}{5^2}-\frac{1}{6^2}\right) = \frac{11}{900} R
\end{equation*}

La energía es:

\begin{equation*}
E = \frac{hc}{\lambda_{5\rightarrow 6}}  =2.7 \times 10^{-20} \text{ J} = 0.17 \text{ eV}
\end{equation*}

\noindent\rule{16.5cm}{0.4pt}
\\



\begin{center}
	\textbf{Fórmulas útiles}
\end{center}



Radio de Bohr:

\begin{equation*}
a_0 = \frac{\hbar}{\mu c \alpha}
\end{equation*}

Radio enésimo para un átomo parecido al hidrógeno:

\begin{equation*}
a_n = a_0 n^2
\end{equation*}

Energía enésima para un átomo parecido al hidrógeno:
\begin{equation}
E_n = -\frac{(k_e e^2)^2 \mu }{2 \hbar^2 n^2}
\end{equation}

Formula de Rydberg:

\begin{equation*}
\frac{1}{\lambda} = R\left(\frac{1}{n_f^2}-\frac{1}{n_i^2}\right)
\end{equation*}

Donde $R = 0.010972$ nm$^{-1}$.


\end{document}