  \documentclass[12pt]{article}
 
\usepackage[margin=1in]{geometry}
\usepackage{amsmath,amsthm,amssymb}
\usepackage[spanish]{babel}
\decimalpoint
\usepackage[utf8]{inputenc}
\usepackage{enumitem, kantlipsum}
\usepackage{graphicx}
\newcommand{\angstrom}{\mbox{\normalfont\AA}}
\setlength{\parindent}{0cm} 

\begin{document}
 
\begin{center}
\Large \textbf{C.Física Moderna: Taller 7}\\
\normalsize \textbf{Espectros atómicos y modelo de Bohr}
\end{center}
 
  

\section{Transiciones del átomo de hidrógeno}

Se tienen las siguientes cuatro transiciones del átomo de hidrógeno:

\begin{enumerate}[label=\alph*)]
	\item $n_i= 2\quad n_f= 5$
	\item $n_i=5 \quad n_f= 3$
	\item $n_i=7 \quad n_f=4$
	\item $n_i=4 \quad n_f=7$
\end{enumerate}

\begin{enumerate}
	\item  ¿Para cuáles transiciones perdió energía el átomo?
	\item  ¿Para cuál transición se emite los fotones con menor longitud de onda? 
	\item ¿Para cuál transición gana mayor
	energía el átomo?
\end{enumerate}
\noindent\rule{16.5cm}{0.4pt}





\section{Serie de Lyman para el hidrógeno}

Calcule la longitud más grande y pequeña en la serie de Lyman para el hidrógeno. Diga la transición electrónica
que da lugar a cada una. ¿Hay lineas espectrales de Lyman que están en el espectro visible?.
	


\noindent\rule{16.5cm}{0.4pt}


\section{Positronio}

El Positronio es un átomo similar al de hidrógeno que consiste de un positrón y un electrón dando vueltas el uno
alrededor del otro. Usando el modelo de Bohr:

\begin{enumerate}
	\item Encuentre los radios permitidos (relativos al centro de masa) $a_{n,\text{Ps}}$ y las
	energías permitidas del sistema $E_{n,\text{Ps}}$.
	\item Calcule la razón de la energía enésima del Positronio a la del átomo de hidrógeno ($E_{n,H}/E_{n,\text{Ps}}$).
\end{enumerate}

\textbf{Ayuda: El radio de Bohr y la mínima energía  para el átomo de hidrógeno son:} $a_{o,H} = 0.53 \angstrom$ y  $E_{1,H} = -13.6 \text{ eV}$.

\noindent\rule{16.5cm}{0.4pt}

\section{Energía para una transición}

Calcule la energía de un foton que podría causar una transición electrónica de:


\begin{enumerate}[label=\alph*)]
	\item $n_i = 4$ a $n_f = 5$ 
	\item $n_i = 5$ a $n_f = 6$
\end{enumerate}



\noindent\rule{16.5cm}{0.4pt}
\\
\begin{center}
\textbf{Fórmulas útiles}
\end{center}



Radio de Bohr:

\begin{equation*}
a_0 = \frac{\hbar}{\mu c \alpha}
\end{equation*}

Radio enésimo para un átomo parecido al hidrógeno:

\begin{equation*}
a_n = a_0 n^2
\end{equation*}

Energía enésima para un átomo parecido al hidrógeno:
\begin{equation}
E_n = -\frac{(k_e e^2)^2 \mu }{2 \hbar^2 n^2}
\end{equation}

Formula de Rydberg:

\begin{equation*}
\frac{1}{\lambda} = R\left(\frac{1}{n_f^2}-\frac{1}{n_i^2}\right)
\end{equation*}

Donde $R = 0.010972$ nm$^{-1}$.

\end{document}