\documentclass{article}
\usepackage{braket}
\usepackage[latin1]{inputenc}
\usepackage{amsfonts}
\usepackage{amsthm}
\usepackage{amsmath}
\usepackage{mathrsfs}
\usepackage{enumitem}
\usepackage[pdftex]{color,graphicx}
\usepackage{hyperref}
\usepackage{listings}
\usepackage{calligra}
\usepackage{algpseudocode} 
\DeclareFontShape{T1}{calligra}{m}{n}{<->s*[2.2]callig15}{}
\newcommand{\scripty}[1]{\ensuremath{\mathcalligra{#1}}}
\setlength{\oddsidemargin}{0cm}
\setlength{\textwidth}{490pt}
\setlength{\topmargin}{-40pt}
\addtolength{\hoffset}{-0.3cm}
\addtolength{\textheight}{4cm}
%	\centering
%	\includegraphics[scale = 0.42]{lcaoderecha}
%	\caption{Eletr\'on ligado solo al n�cleo derecho}
%	\label{fig1}
%	\end{figure}




\begin{document}
	%\tableofcontents
	%\pagebreak
	
	
	
	
	
	
	
	
	
	
	\begin{center}
		\textsc{\LARGE F�sica Moderna}\\[0.2cm]
		\textsc{\LARGE  Soluci�n Taller 9  }\\[0.2cm]
	\end{center}
	

	


\section{Modelo para un �tomo como un electr�n en una caja}
Cierto �tomo requiere 3 eV de energ�a para excitar un electr�n desde el nivel fundamental
al primer nivel excitado. Modele el �tomo como un electr�n en una caja 1D y calcule el ancho
$L$ de la caja


\begin{center}
	\textbf{Soluci�n}
\end{center}

Se calcula la diferencia de energ�a entre el primer y segundo nivel de energ�a:

\begin{equation}
\Delta E = E_2- E_1 = (4-1)\frac{ \pi^2 \hbar^2}{2mL^2} 
\end{equation}

Se desepeja $L$:

\begin{equation*}
L = \sqrt{\frac{ 3\pi^2 \hbar^2}{2m\Delta E}} = 6.14 \AA
\end{equation*}


\noindent\rule{16.5cm}{0.4pt}

\section{Problema inverso: Obtener el potencial a partir de la funci�n de onda}
En una regi�n del espacio, una part�cula tiene una funci�n de onda dada por:

\begin{equation*}
\psi(x) = A \exp\left(-\frac{x^2}{2L^2}\right)
\end{equation*}
Y energ�a dada por:

\begin{equation*}
E =  \frac{\hbar^2}{2mL^2}
\end{equation*}

Donde $L$ representa una longitud.

\begin{enumerate}
	\item  Calcule la energ�a potencial como una funci�n de $x$.
	\item  Esboce un gr�fico de $V(x)$ vs $x$. 
\end{enumerate}

\begin{center}
	\textbf{Soluci�n}
\end{center}

2.1 Se despeja de la ecuaci�n de Schrodinger el potencial:

\begin{equation*}
V(x) = E + \frac{\hbar^2}{2m \psi(x)} \frac{d^2\psi(x)}{dx^2}
\end{equation*}

Al calcular la derivada y reemplazar el valor de la energ�a se obtiene:

\begin{equation*}
V(x) = \frac{\hbar^2 x^2}{2m L^4}
\end{equation*}

2.2 La gr�fica es una par�bola con concavidad positiva.

\noindent\rule{16.5cm}{0.4pt}

\section{Electr�n en un pozo cuadrado infinito 1D}
Un electr�n que se mueve en un pozo cuadrado infinito unidimensional est� atrapado en el
estado $n = 5$.

\begin{enumerate}
	\item  Calcule la probabilidad de encontrar el electr�n entre $x = 0.2L$ y $x = 0.4L$.
	\item Calcule la probabilidad de hallar el electr�n dentro de un ancho $\Delta x = L/100$ en $x = L/2$. Es decir calcule la probabilidad de encontrar el electr�n en el intervalo  $(x-\frac{\Delta x}{2},x+\frac{\Delta x}{2})$
\end{enumerate}

\begin{center}
	\textbf{Soluci�n}
\end{center}

3.1  La probabilidad se calcula haciendo la siguiente integral:

\begin{equation*}
P(0.2L<x<0.4L) = \int_{0.2L}^{0.4L} |\psi_5(x)|^2 dx = 0.2
\end{equation*}

3.2 Se repite la integral anterior pero con limites distintos:

\begin{equation*}
P(0.01L-\frac{\Delta x}{2}<x<0.01L+\frac{\Delta x}{2}) = \int_{0.01L-\frac{\Delta x}{2}}^{0.01L+\frac{\Delta x}{2}} |\psi_5(x)|^2 dx = 0.01 + \frac{1}{5\pi} \sin(0.05\pi) \approx 0.02
\end{equation*}



\noindent\rule{16.5cm}{0.4pt}

\section{Electr�n en un potencial constante}

Se tiene una part�cula de masa $m$ en una regi�n donde la energ�a potencial es una constante $V_0 < E$.

\begin{enumerate}
	\item  Demuestre que una soluci�n de la ecuaci�n de Schr�dinger
	independiente del tiempo para esa part�cula esta dada por:
	\begin{equation*}
	\psi(x) = Ae^{ikx}
	\end{equation*}
	\item Encuentre una expresi�n para $k$.
	\item Calcule la longitud de onda de DeBroglie para el electr�n.
\end{enumerate}


\begin{center}
	\textbf{Soluci�n}
\end{center}

4.1 Se calcula la segunda derivada:

\begin{equation*}
\frac{d^2\psi(x)}{dx^2} =  - k^2 \psi(x)
\end{equation*}

Reemplazando en la ecuaci�n de Schrodinger vemos que si se cumple la siguiente igualdad, $\phi$ es soluci�n:

\begin{equation*}
\frac{\hbar^2 k^2}{2m} + V_0 = E
\end{equation*}

Note que $\hbar^2 k^2/(2m)$ hace el papel de energ�a cin�tica y por lo tanto $\hbar k$ de momento.\\ 

4.2 De la igualdad despejamos $k$:

\begin{equation*}
k = \frac{1}{\hbar} \sqrt{2m (E-V_0)}
\end{equation*}

4.3 El n�mero de onda lo podemos asociar con una longitud de onda:

\begin{equation*}
\lambda = \frac{2 \pi}{k} = \frac{h}{\sqrt{2m (E-V_0)}}
\end{equation*}

\noindent\rule{16.5cm}{0.4pt}
	
\end{document}