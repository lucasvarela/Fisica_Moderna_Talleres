\documentclass{article}
\usepackage{braket}
\usepackage[latin1]{inputenc}
\usepackage{amsfonts}
\usepackage{amsthm}
\usepackage{amsmath}
\usepackage{mathrsfs}
\usepackage{enumitem}
\usepackage[pdftex]{color,graphicx}
\usepackage{hyperref}
\usepackage{listings}
\usepackage{calligra}
\usepackage{algpseudocode} 
\DeclareFontShape{T1}{calligra}{m}{n}{<->s*[2.2]callig15}{}
\newcommand{\scripty}[1]{\ensuremath{\mathcalligra{#1}}}
\setlength{\oddsidemargin}{0cm}
\setlength{\textwidth}{490pt}
\setlength{\topmargin}{-40pt}
\addtolength{\hoffset}{-0.3cm}
\addtolength{\textheight}{4cm}
\usepackage{subcaption}
\usepackage{amssymb}

%\begin{figure}[H]
%	\centering
%	\includegraphics[scale = 0.42]{lcaoderecha}
%	\caption{Eletr\'on ligado solo al n�cleo derecho}
%	\label{fig1}
%	\end{figure}




\begin{document}
	%\tableofcontents
	%\pagebreak
\begin{center}
	\Large \textbf{C.F�sica Moderna: Taller 9}\\
	\normalsize \textbf{Introducci�n a la ecuaci�n de Schr�dinger}
\end{center}


	
\section{Modelo para un �tomo como un electr�n en una caja}
	Cierto �tomo requiere 3 eV de energ�a para excitar un electr�n desde el nivel fundamental
	al primer nivel excitado. Modele el �tomo como un electr�n en una caja 1D y calcule el ancho
	$L$ de la caja
	
\noindent\rule{16.5cm}{0.4pt}
	
\section{Problema inverso: Obtener el potencial a partir de la funci�n de onda}
	 En una regi�n del espacio, una part�cula tiene una funci�n de onda dada por:
	 
	 \begin{equation*}
	 \psi(x) = A \exp\left(-\frac{x^2}{2L^2}\right)
	 \end{equation*}
	 Y energ�a dada por:
	 
	 \begin{equation*}
	 E =  \frac{\hbar^2}{2mL^2}
	 \end{equation*}
	
	Donde $L$ representa una longitud.
	
	\begin{enumerate}
		\item  Calcule la energ�a potencial como una funci�n de $x$.
		\item  Esboce un gr�fico de $V(x)$ vs $x$. 
	\end{enumerate}

\noindent\rule{16.5cm}{0.4pt}

\section{Electr�n en un pozo cuadrado infinito 1D}
 Un electr�n que se mueve en un pozo cuadrado infinito unidimensional est� atrapado en el
estado $n = 5$.

\begin{enumerate}
	\item  Calcule la probabilidad de encontrar el electr�n entre $x = 0.2L$ y $x = 0.4L$.
	\item Calcule la probabilidad de hallar el electr�n dentro de un ancho $\Delta x = L/100$ en $x = L/2$. Es decir calcule la probabilidad de encontrar el electr�n en el intervalo  $(x-\frac{\Delta x}{2},x+\frac{\Delta x}{2})$
	\end{enumerate}

\noindent\rule{16.5cm}{0.4pt}

\section{Electr�n en un potencial constante}

Se tiene una part�cula de masa $m$ en una regi�n donde la energ�a potencial es una constante $V_0 < E$.

\begin{enumerate}
	\item  Demuestre que una soluci�n de la ecuaci�n de Schr�dinger
	independiente del tiempo para esa part�cula esta dada por:
	\begin{equation*}
	\psi(x) = Ae^{ikx}
	\end{equation*}
	\item Encuentre una expresi�n para $k$.
	\item Calcule la longitud de onda de DeBroglie para el electr�n.
\end{enumerate}


\noindent\rule{16.5cm}{0.4pt}
\pagebreak
\begin{center}
	\textbf{F�rmulas �tiles}
\end{center}

Las energ�as permitidas en una caja (pozo de potencial infinito) son
\begin{equation*}
E_n = \frac{n^2 \pi^2 \hbar^2}{2mL^2}
\end{equation*}

Ecuaci�n de Schr�dinger:

\begin{equation*}
\left(-\frac{\hbar^2}{2m} \frac{d^2}{dx^2}   + V\right) \psi(x) = E \psi(x) 
\end{equation*}


Las funciones de onda para los estados posibles en un pozo de potencial infinito 1D	 son:

\begin{equation*}
\psi_n(x) = \sqrt{\frac{2}{L}} \sin\left(\frac{n \pi x}{L}\right)
\end{equation*}

\end{document}