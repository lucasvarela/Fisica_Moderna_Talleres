\documentclass{article}
\usepackage{braket}
\usepackage[latin1]{inputenc}
\usepackage{amsfonts}
\usepackage{amsthm}
\usepackage{amsmath}
\usepackage{mathrsfs}
\usepackage{enumitem}
\usepackage[pdftex]{color,graphicx}
\usepackage{hyperref}
\usepackage{listings}
\usepackage{calligra}
\usepackage{algpseudocode} 
\DeclareFontShape{T1}{calligra}{m}{n}{<->s*[2.2]callig15}{}
\newcommand{\scripty}[1]{\ensuremath{\mathcalligra{#1}}}
\setlength{\oddsidemargin}{0cm}
\setlength{\textwidth}{490pt}
\setlength{\topmargin}{-40pt}
\addtolength{\hoffset}{-0.3cm}
\addtolength{\textheight}{4cm}
\usepackage{subcaption}
\usepackage{amssymb}

%\begin{figure}[H]
%	\centering
%	\includegraphics[scale = 0.42]{lcaoderecha}
%	\caption{Eletr\'on ligado solo al n�cleo derecho}
%	\label{fig1}
%	\end{figure}




\begin{document}
	%\tableofcontents
	%\pagebreak
\begin{center}
	\Large \textbf{C.F�sica Moderna: Taller 9}\\
	\normalsize \textbf{Longitud de onda de De-Broglie, experimento de Davisson-Germer y principio de incertidumbre}
\end{center}


	
\section{Electr�n libre}
	
Un electr�n tiene velocidad $v_0$ NO relativista.	Muestre que la velocidad de grupo del electr�n es $v_g = v_0$.
	
\noindent\rule{16.5cm}{0.4pt}
	
\section{Electr�n libre relativista}
	
	La relaci�n de dispersi�n para un electr�n libre relativista est� dada por:
	
	\begin{equation}
		\omega(k) = \sqrt{c^2k^2 + (m_ec^2/\hbar)^2}
	\end{equation}
	
	Obtenga una expresi�n para la velocidad de grupo $v_g$ y de fase $v_p$. Muestre que el producto de dichas cantidades es constante. Del resultado comente que le pasa a $v_g$ si $v_p>c$.
	
\noindent\rule{16.5cm}{0.4pt}
		
\section{Humano Extra-dimensional}
	
	 Nicolas Jaar es un humano que vive en un universo donde $h = 2 \pi J\cdot s$. �l tiene una masa de 2.0 kg y inicialmente se sabe que se encuentra en una regi�n de 1.0 m de largo. 
	
	\begin{enumerate}
		\item �Cu�l es la m�nima incertidumbre de su velocidad?
		\item Asumiendo que la incertidumbre de su velocidad va a durar 5.0 s, determine la incertidumbre de su posici�n luego de ese tiempo.
	\end{enumerate}

\noindent\rule{16.5cm}{0.4pt}

\section{Detectores gamma}
	
	Un n�cleo excitado con una vida media de 0.1 ns emite un rayo $\gamma$ con energ�a de 2.00 MeV. Si los mejores detectores gamma pueden medir energ�as de $\pm $ 5 eV, �pueden estos detectores medir el ancho de energ�a ($\Delta E$) para dicha emisi�n?
	
\noindent\rule{16.5cm}{0.4pt}
	

\section{Grosor natural de las lineas espectrales}

Las lineas de un espectro tienen un grosor. Este grosor en varios casos es causado por el efecto doppler. Sin embargo existe tambi�n un grosor asociado a que las transiciones ocurren en un tiempo finito, lo cual da una incertidumbre intr�nseca. Para transiciones at�micas la vida media para que ocurra una transici�n es $\tau = 10^{-8}s$.

\begin{enumerate}
	\item Calcule la incertidumbre de energ�a para una transici�n. 
	\item La incertidumbre de la longitud de onda $\Delta \lambda$ de una transici�n de un estado excitado $E$ al fundamental $E_1$ es:
	$$ \frac{\Delta \lambda}{\lambda} \approx \frac{\Delta E}{E-E_1}$$
	Calculela para la transici�n 2$\rightarrow$1 del �tomo de hidr�geno.
\end{enumerate}



	
\end{document}